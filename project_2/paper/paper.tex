\documentclass{article}


\usepackage{arxiv}

\usepackage[utf8]{inputenc} % allow utf-8 input
\usepackage[T1]{fontenc}    % use 8-bit T1 fonts
\usepackage{hyperref}       % hyperlinks
\usepackage{url}            % simple URL typesetting
\usepackage{booktabs}       % professional-quality tables
\usepackage{amsfonts}       % blackboard math symbols
\usepackage{nicefrac}       % compact symbols for 1/2, etc.
\usepackage{microtype}      % microtypography
\usepackage{lipsum}		% Can be removed after putting your text content
\usepackage{textgreek}
\usepackage{amsmath}
\usepackage{algorithm}
\usepackage{algpseudocode}
\usepackage{algpascal}
\usepackage{graphicx}
\usepackage[
backend=biber,
style=numeric,
sorting=ynt,
citestyle=authoryear
]{biblatex}
\addbibresource{paper.bib}

%\graphicspath{ {./images/} }

\title{Playing Lunar Lander with Deep Reinforcement Learning}

\date{June 28th, 2019}

\author{
Carlos Souza\thanks{Latest Git hash: INCLUDE LATEST GITHASH}\\
%Department of Computer Science\\
Georgia Institute of Technology\\
São Paulo, SP, Brazil \\
\texttt{souza@gatech.edu} \\
}

\begin{document}
    \maketitle

    \begin{abstract}
        \lipsum[1]
    \end{abstract}


    % keywords can be removed
    \keywords{Reinforcement learning \and Deep Reinforcement Learning \and Deep Q-Networks}
    %\githash{DO NOT FORGET TO INCLUDE LATEST GIT HASH}


    \section{Introduction}
    \label{sec:introduction}
    Reinforcement learning (RL) theory provides mathematical background and classic algorithms to enable agents to learn optimally control an environment.
    However, to successfuly apply those algorithms in problems with real-world complexity, several challenges must be tackled, must notably i) how to derive efficient representations of the environment from high-dimensional sensory inputs, and ii) how to use these to generalize past experience to new situation (\cite{dqn}).

    Recent advances in deep neural networks originated a new type of agent, known as Deep Q-Network agent, capable of overcoming these challenges, which enabled RL algorithms to perform effectively.
    All the many recent successes in applying RL to complex sequential decision-making problems were kick-started by the Deep Q-Network algorithm (DQN; \cite{dqn}).
    Since then, many extensions have been proposed to improve its stability and/or speed.
Double DQN (DDQN; \cite{ddqn}) addresses the problem of overestimation of action values, by decomposing the max operation in the target into action selection and action evaluation.
    The Dueling Network architecture (Dueling DDQN; \cite{dueling}) better generalize learning accross actions by proposing an architecture that explicitly separates the representation of state values and (state-dependent) action advantages.
    Prioritized Experience Replay (PER; \cite{per}) improves data efficiency, by replaying more of- ten transitions from which there is more to learn.






    \printbibliography

\end{document}
